\documentclass{article}
\usepackage{graphicx} % Required for inserting images
\usepackage{amsmath}
\usepackage{amssymb}
\title{Problemas semana 2}
\author{Mates y más}
\date{Junio 16 2025}

\begin{document}

\maketitle

\section{Problema 1}
¿De cuántas maneras se pueden acomodar siete canicas blancas y cinco negras en una línea de tal manera que no haya dos canicas negras vecinas?

\section{Problema 2}

    Si $a$ y $b$ son enteros positivos, demostrar que $11a+2b$ es múltiplo de $19$ si y sólo si lo es $18a+5b$.

\section{Problema 3}

    Se dan dos circunferencias externamente tangentes con radios diferentes. Sus tangentes comunes forman un triángulo. Halla el área de este triángulo en función de los radios de las dos circunferencias.

\section{Problema 4}

    ¿De cuántas maneras se pueden seleccionar ocho enteros $a_1,a_2, ... ,a_8$, no necesariamente distintos, tales que $1 \le a_1 \le ... \le a_8 \le 8$?

\section{Problema 5}

    Considera dos puntos fijos $B,C$ en una circunferencia $w$. Encuentra el locus de los incentros de todos los triángulos $ABC$ cuando el punto $A$ se mueve sobre $w$.

\section{Problema 6}

    Considera dos puntos fijos $B,C$ en una circunferencia $w$. Encuentra el locus de los incentros de todos los triángulos $ABC$ cuando el punto $A$ se mueve sobre $w$.

\section{Problema 7}

    Dos subconjuntos disjuntos del conjunto $\{1,2, ... ,m\}$ tienen la misma suma de elementos. Demostrar que cada uno de los subconjuntos $A,B$ tiene menos de $m\sqrt{2}$ elementos.

\section{Problema 8}

    Calcula el volumen de un octaedro regular circunscrito a una esfera de radio $1$.

    
\section{Problemas extra}
\subsection{Problema 1 extra}

Determine todas las ternas de números enteros positivos \( (a, b, c) \) que satisfacen el siguiente sistema de ecuaciones:

\begin{align*}
    a\sqrt{b} - c &= a, \\
    b\sqrt{c} - a &= b, \\
    c\sqrt{a} - b &= c.
\end{align*}

\subsection{Problema 2 extra}

Encuentra todos los números naturales $n$ para los que se cumple 

\[
    \sigma(n) + \varphi(n) = 2n
\]
\textbf{Nota:} Para \( m \in \mathbb{N}^{*} \), se define:

\begin{itemize}
    \item \( \sigma(m) \): la suma de los divisores positivos de \( m \).
    \item \( \varphi(m) = \#\left\{ a \in \mathbb{N} \;\middle|\; 1 \leq a \leq m,\ \gcd(a, m) = 1 \right\} \): la cantidad de enteros entre 1 y \( m \) que son coprimos con \( m \).
\end{itemize}

\end{document}
